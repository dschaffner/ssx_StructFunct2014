%%%%%%%%%%%%%%%%%%%%%%%%%%%%%%%%%%%%%%%%%%%%%%%%%%%%%%%%%%%%%%%%%%%%%%%%
%    INSTITUTE OF PHYSICS PUBLISHING                                   %
%                                                                      %
%   `Preparing an article for publication in an Institute of Physics   %
%    Publishing journal using LaTeX'                                   %
%                                                                      %
%    LaTeX source code `ioplau2e.tex' used to generate `author         %
%    guidelines', the documentation explaining and demonstrating use   %
%    of the Institute of Physics Publishing LaTeX preprint files       %
%    `iopart.cls, iopart12.clo and iopart10.clo'.                      %
%                                                                      %
%    `ioplau2e.tex' itself uses LaTeX with `iopart.cls'                %
%                                                                      %
%%%%%%%%%%%%%%%%%%%%%%%%%%%%%%%%%%
%
%
% First we have a character check
%
% ! exclamation mark    " double quote  
% # hash                ` opening quote (grave)
% & ampersand           ' closing quote (acute)
% $ dollar              % percent       
% ( open parenthesis    ) close paren.  
% - hyphen              = equals sign
% | vertical bar        ~ tilde         
% @ at sign             _ underscore
% { open curly brace    } close curly   
% [ open square         ] close square bracket
% + plus sign           ; semi-colon    
% * asterisk            : colon
% < open angle bracket  > close angle   
% , comma               . full stop
% ? question mark       / forward slash 
% \ backslash           ^ circumflex
%
% ABCDEFGHIJKLMNOPQRSTUVWXYZ 
% abcdefghijklmnopqrstuvwxyz 
% 1234567890
%
%%%%%%%%%%%%%%%%%%%%%%%%%%%%%%%%%%%%%%%%%%%%%%%%%%%%%%%%%%%%%%%%%%%
%

%AIP Reprint Class%%%%%%%%%%%%%%%%%%%%%%%%%%%%%%%%%%%%%%%%%%%%%%%%%%%%%%%%%%%%%%%%%%%%%%%%%%%%%
%\documentclass[aps,prl,amsmath,amssymb,reprint,superscriptaddress]{revtex4-1} %preprint version
%\usepackage{graphicx}% Include figure files
%\usepackage{dcolumn}% Align table columns on decimal point
%\usepackage{bm}% bold math
%\usepackage{epstopdf}

%    \renewcommand{\topfraction}{0.9}    % max fraction of floats at top
%    \renewcommand{\bottomfraction}{0.8}    % max fraction of floats at bottom
%    \setcounter{topnumber}{2}
%    \setcounter{bottomnumber}{2}
%    \setcounter{totalnumber}{4}     % 2 may work better
%    \setcounter{dbltopnumber}{2}    % for 2-column pages
%    \renewcommand{\dbltopfraction}{0.9}    % fit big float above 2-col. text
%    \renewcommand{\textfraction}{0.07}    % allow minimal text w. figs
%    \renewcommand{\floatpagefraction}{0.7}    % require fuller float pages
%    \renewcommand{\dblfloatpagefraction}{0.7}    % require fuller float pages
%    \setlength{\abovecaptionskip}{5pt}
%    \setlength{\belowcaptionskip}{5pt}
%    \setlength{\parskip}{0pt}
%    \setlength{\textfloatsep}{5pt} 

%%%%%%%%%%%%%%%%%%%%%%%%%%%%%%%%%%%%%%%%%%%%%%%%%%%%%%%%%%%%%%%%%%%%%%%%%%%%%%%%%%%%%%%%%%%%%%%%%%

%IOP preprint class %%%%%%%%%%%%%%%%%%%%%%%%%%%%%%%%%%%%%%%%%%%%%%%%%%%%%%%%%%%%%%%%%%%%%%%%%%%%%%
%\documentclass[12pt]{iopart}
%\newcommand{\gguide}{{\it Preparing graphics for IOP journals}}
%Uncomment next line if AMS fonts required
%\usepackage{iopams}
%\usepackage{graphicx}
%\usepackage{epstopdf}  
%%%%%%%%%%%%%%%%%%%%%%%%%%%%%%%%%%%%%%%%%%%%%%%%%%%%%%%%%%%%%%%%%%%%%%%%%%%%%%%%%%%%%%%%%%%%%%%%%%
%Slava's inserts %%%%%%%%%%%%%%%%%%%%%%%%%%%%%%%%%%%%%%%%%%%%%%%%%%%%%%%%%%%%%%%%%%%%%%%%%%%%%%
%\usepackage{amsfonts}
%\usepackage{amssymb}

%\newcommand{\ptt}[1]{\frac{\partial#1}{\partial t}}
%\newcommand{\vvec}{\mathbf{v}}
%\newcommand{\Bvec}{\mathbf{B}}
%\newcommand{\Evec}{\mathbf{E}}
%\newcommand{\Jvec}{\mathbf{J}}
%\newcommand{\Avec}{\mathbf{A}}
%%%%%%%%%%%%%%%%%%%%%%%%%%%%%%%%%%%%%%%%%%%%%%%%%%%%%%%%%%%%%%%%%%%%%%%%%%%%%%%%%%%%%%%%%%%%%%%%%%

\documentclass[aps,prl,amsmath,amssymb,reprint,superscriptaddress]{revtex4-1} %preprint version
\usepackage{graphicx}% Include figure files
\usepackage{dcolumn}% Align table columns on decimal point
\usepackage{bm}% bold math

    \renewcommand{\topfraction}{0.9}    % max fraction of floats at top
    \renewcommand{\bottomfraction}{0.8}    % max fraction of floats at bottom
    \setcounter{topnumber}{2}
    \setcounter{bottomnumber}{2}
    \setcounter{totalnumber}{4}     % 2 may work better
    \setcounter{dbltopnumber}{2}    % for 2-column pages
    \renewcommand{\dbltopfraction}{0.9}    % fit big float above 2-col. text
    \renewcommand{\textfraction}{0.07}    % allow minimal text w. figs
    \renewcommand{\floatpagefraction}{0.7}    % require fuller float pages
    \renewcommand{\dblfloatpagefraction}{0.7}    % require fuller float pages
    \setlength{\abovecaptionskip}{5pt}
    \setlength{\belowcaptionskip}{5pt}
    \setlength{\parskip}{0pt}
    \setlength{\textfloatsep}{5pt} 

\begin{document}
\title{Scaling Differences between Inertial and Dissipation Range Turbulence Observed through Temporal and Spatial Structure Function Analysis of Laboratory Based MHD Experiment}

\author{D.A. Schaffner}
\affiliation{Swarthmore College, Swarthmore, PA, USA}
\author{M.R. Brown}
\affiliation{Swarthmore College, Swarthmore, PA, USA}
%\author{V.S. Lukin}
%\affiliation{Space Science Division, Naval Research Laboratory, Washington, DC, USA}

\date{\today}
\begin{abstract}
The nature of MHD turbulence is analyzed using temporal and spatial structure function analysis to understand the intermittent nature of magnetic field fluctuations in an MHD plasma wind-tunnel.
%The nature of MHD turbulence is analyzed through both temporal and spatial magnetic fluctuation spectra. A magnetically turbulent plasma is produced in the MHD wind-tunnel configuration of the Swarthmore Spheromak Experiment (SSX). The power of magnetic fluctuations is projected into directions perpendicular and parallel to a local mean field; the ratio of these quantities shows the presence of variance anisotropy which varies as a function of frequency. Comparison amongst magnetic, velocity, and density spectra are also made, demonstrating that the energy of the turbulence observed is primarily seeded by magnetic fields created during plasma production. Direct spatial spectra are constructed using multi-channel diagnostics and are used to compare to frequency spectra converted to spatial scales using the Taylor Hypothesis. Evidence for the observation of dissipation due to ion inertial length scale physics is also discussed as well as the role laboratory experiment can play in understanding turbulence typically studied in space settings such as the solar wind. Finally, all turbulence results are shown to compare fairly well to a Hall-MHD simulation of the experiment. 
\end{abstract}

\maketitle

\section{Introduction}

This paper presents the results of a thorough intermittency analysis of the fluctuating magnetic fields in the Swarthmore Spheromak Experiment (SSX) plasma through the use of structure functions and probability distribution functions (PDFs) of increments of both temporal and spatial measurements. The primary observation from this analysis is that temporal regions of the magnetic fluctuation data shown to be consistent with dissipation range turbulence~\cite{schaffner2014c} exhibit structure function scaling that indicates self-similarity in the turbulence structure, whereas temporal regions for inertial range turbulence do not exhibit this self-similarity. These results are discussed in the context of multifractal versus monofractal structure function scaling behavior of turbulence~\cite{frisch1995,marsch1997}, where a lack of self-similarity can be attributed to the multifractal nature of the structure function scaling. Similar distinctions between inertial range and dissipation range scaling is made in recent solar wind observations~\cite{kiyani2013} suggesting that physical mechanisms underlying the scaling in each regime may be universal between solar wind and laboratory based MHD turbulence. Furthermore, the relatively fast transition from multifractal to monofractal scaling suggests a global dissipation mechanism at work, such as the generation of current sheets at the ion inertial scale length~\cite{kiyani2009,kiyani2010}.

Trends in scaling with helicity injection are also explored. Previous work~\cite{schaffner2014b} demonstrated that increased injected helicity in the SSX plasma system resulted in an increased intermittency of the raw $dB/dt(t)$ signal. Results reported here show that the $B(t)$ formed from time integration of the raw $dB/dt$ signal demonstrate this same trend. The scaling behavior of the structure functions differ with helicity depending on being in the dissipation versus the inertial range. Inertial range scaling appears to vary more widely with helicity than the dissipation range. This indicates that the effect of the helicity injection on intermittency has an affect only on structures generated in the inertial range, or on the physical mechanism which does the generating. Conversely, the mechanism that governs intermittency in the dissipation range is not strongly affected by the overall helicity content of the plasma.

Structure function analysis has typically been utilized to extend spectral or correlation analysis of turbulent fluctuations to higher-order statistics, particularly in scenarios where Gaussian and self-similar properties appear to break down. Examined first in hydrodynamics~\cite{frisch1995}, the use of the technique has been expanded to magnetohydrodynamic (MHD) fluids including the solar wind velocity fluctuations~\cite{burlaga1991}, solar wind magnetic fluctuations~\cite{burlaga1992} and magnetospheric plasmas~\cite{consolini1996,hnat2003}. The work has led to the development of models which attempt to reconcile the observation of intermittency, non-self-similar statistics in turbulence fluctuations with the 1941 Kolomogorov turbulence model which implies a self-similar fluctuation structure which can be described in terms of a monofractal scaling exponent~\cite{kolmogorov1941,frisch1995}. Such multifractal models relax global scale invariance to a local scale invariance. Physically, these models assert modifications as to how energy is distributed to smaller scales from larger scales either through differences in the space-filling nature of the turbulence, such as the random $\beta$-model~\cite{frisch1978} or through variations in the energy transfer rate between from larger to smaller scales such as the log-normal model~\cite{kolmogorov1962}, the $p$-model~\cite{meneveau1987} and the She-Leveque model~\cite{she1994,dubrulle1994}. Multifractal scaling models can make predictions for dissipation scaling, in particular addressing a reduction in scaling exponents by viscosity~\cite{frisch1991,chevillard2005}. Since these models were developed with isotropic and homogeneous turbulence of conventional neutral fluids, extending intermittency models to MHD turbulence has added complications. MHD turbulence is inherently anisotropic---due to the symmetry breaking caused by embedded magnetic fields---in addition to having both fluctuations, energy transfer and dissipation channels for both magnetic and velocity fields. While hydrodynamic intermittency models have been directly applied to MHD turbulence~\cite{burlaga1991,pagel2002}, modified models pertaining explicitly to MHD have also been developed~\cite{carbone1993,muller2000,biskamp2000,boldyrev2002,cho2003}. Finally, structure function analysis can be of particular use for extracting signatures of dissipative turbulence~\cite{frisch1991,cho2003,chevillard2005,alexandrova2008,kiyani2009,kiyani2013}.

The structure functions and PDFs are constructed by taking differences or increments at varying time scale separations. In this analysis, the increments of magnetic measurements are constructed in three ways; first, the increments of each orthogonal magnetic component ($\Delta B_r$,$\Delta B_{\theta}$,and $\Delta B_z$) are examined separately. Then, vector magnitude is created from the vector sum of the three components at each time step and differences between the magnitudes ($\Delta |B|$) at each time step are used for the analysis. Lastly, the magnitude of the vector difference between time points is used as the increment ($|\Delta B|$). The results for these five different forms are generally similar to one another suggesting that the intermittent characteristics of the fluctuations are not strongly anisotropic. The paper also demonstrates that the trends for the fluctuations in the magnetic field are consistent with that found fluctuations in $dB/dt(t)$ which were reported in previous work~\cite{schaffner2014a,schaffner2014b}.

\section{Experiment}\label{sec:experiment}

Magnetohydrodynamic turbulent fluctuations are generated inside the wind tunnel configuration of the Swarthmore Spheromak Experiment (SSX) using a plasma gun source inside a flux-conserving copper boundary. This copper cylinder is 15.5cm in diameter and 86cm long and situated inside a highly-evacuated chamber ($\approx 1 \times 10^{-8}$ Torr). The details of the gun source and production of magnetic fluctuations have been previously reported~\cite{schaffner2014c,brown14,brown15a}. The data analyzed in this paper is of magnetic fluctuations using an array of sixteen magnetic pickup channels. Each channel consists of a single-loop of magnetic wire, 3mm in diameter. Three loops are oriented in each cylindrical coordinate direction ($B_{r},B_{\theta},B_{z}$) and each triplet is separated by 0.46cm spanning from about 1cm off the cylindrical axis to the edge of the cylinder boundary. The injected helicity of the plasma is scanned from $0$ to $7\times 10^{-5}~Wb^{2}$ which corresponds to a scan of the amount of flux provided to the gun core--from $0$ to $1.5~mWb$~\cite{schaffner2014b}.

\section{Analysis Techniques}\label{sec:analysis}

The structure of the turbulent plasma is analyzed here through an evaluation of its intermittent character; the primary method for quantifying this characteristic is through the construction of probability distribution functions (PDFs) of increments. The increments can be in terms of a spatial division or a temporal division. In the context of this experiment, the spatial increments are in units of separation distance between probe locations ($\Delta r_{min} = 0.46cm$), while the temporal increments are in units of sampling cadence ($\Delta \tau_{min} = 154ns$) which corresponds to the sampling frequency of 65MHz. Increments are then increased in multiples of these minimum values. Maximum separation values are limited by physical distance or data acquisition time span, as well the ability to generate enough statistics for a valid calculation.

For a given increment, $\Delta x$, the PDF of increments is constructed by computing and histogramming a list of $\Delta B$'s,
\begin{equation}
\Delta B = |B(x_{j}+\Delta x)-B(x_{j})|
\label{eq:increment}
\end{equation}
which can then be normalized to the standard deviation and total count of the histogram.

A single PDF can only describe the nature of the data in terms of its relative distribution with respect to a Gaussian distribution. Large excursions of values in the tail values of the distribution are then indicative of intermittent behavior in the time series signal---i.e. large jumps in values away outside of the standard deviation of the mean. Physically, these excursions can be identified with physical mechanisms in the plasma.

Further insight into the physical nature of the plasma can be gained by comparing these PDFs over a range of scales. This can be accomplished qualitatively by examining how the PDFs themselves change, but can also be quantitatively accomplished by calculating moments of these distributions, also know as structure functions. The structure function can be constructed for a given $\Delta x$ by computing the average of the increments raised to some power, $p$, as in,
\begin{equation}
S^{p}(\Delta x) = \langle|B(x_{j}+\Delta x)-B(x_{j})|^{p}\rangle
\label{eq:structfunc}
\end{equation}
where j indicates the elements summed over. 

Analysis of structure functions can provide a number of features of the fluctuations particularly when modeled as a power-law function,
\begin{equation}
S^{p}(\Delta x) = (\Delta x)^{\zeta(p)}
\label{eq:power-law}
\end{equation}
When plotted logarithmically, $\zeta$ is the slope of the structure function; this slope can indicate the relative level of intermittency compared between two signals. That is, the steeper the slope, the more intermittent (i.e. the larger the ''fat tails``) is the signal. A flat line is the extreme case and indicates a lack of intermittency in the signal. As one would expect, the structure function of a timeseries generated from a normal distribution has a flat structure function. Changes in slope with scale indicates a change in intermittency as a function of scale. It should be noted that this trend is not a perfect predictor of the presence of intermittency however. It can be shown that a time-series of fractional Brownian motion can be constructed with the same structure function slope~\cite{hnat2003}, but fBm produces a Gaussian probability distribution of increments. However, when intermittency is known to be present based on the PDF of increments (as is true for this SSX dataset) then the relationship between slope and degree of intermittency generally holds true.

If $\zeta$ scales as a linear function of $p$ (i.e. $\zeta = Hp$), the system exhibits self-similarity at different scales. In this context, the self-similarity is a qualitative measure of how alike fluctuations appear at different scales. For example, a fractal from chaos theory is a construct that exhibits self-similarity. The constant coefficient,$H$, is called the Hurst exponent as it can be shown that distributions that exhibit this self-similarity can be rescaled using the Hurst exponent as,
\begin{equation}
PDF(\Delta B\Delta x^{-H}) = (\Delta x)^H(PDF(\Delta B,\Delta x)
\label{eq:scaling}
\end{equation}
A non-linear $\zeta(p)$, however, indicates non-self-similar behavior. These distinctions again can be used to elucidate possible physical mechanisms at play in the plasma. 

In particular, it has been shown that differences in self-similarity exist between inertial range and dissipation data in the solar wind~\cite{kiyani2009,kiyani2013}. Using the structure function analysis, the slope of the structure functions of the inertial range were not linear as a function of order, indicating that the inertial range turbulent fluctuations in the solar wind did not indicate self-similarity of turbulent structure. On the other hand, the same analysis for dissipation range structure functions exhibited linear scaling---apparently, the physical mechanism behind the dissipation in the solar wind has a self-similar, or scale-invariant nature.

\section{Temporal and Spatial PDFs of Increments}\label{sec:pdfs}

\begin{figure*}[!htbp]
\centerline{
\includegraphics[width=17cm]{Bmod_pdf_temporal_and_spatial_100313Shots41to80_forStructFuncpaper.png}}
\caption{\label{fig:pdfs} }
\end{figure*}

Representative PDFs of both temporal and spatial magnetic field magnitude ($\Delta |B|$) increments are shown in Figure~\ref{fig:pdfs} with temporal PDFs on the left and spatial PDFs on the right. Each PDFs is scaled to its own standard deviation and total count so that they can be cross compared. The increments chosen for the temporal PDFs are $0.092\mu s$, $0.46\mu s$, and $1.85\mu s$ which correspond to frequencies of $10.8MHz$, $2.2MHz$, and $541kHz$ respectively. These three times also approximately correspond to three different regimes of the frequency spectra, called for lack of more accurate description the dissipation, transition and inertial region. All three PDFs exhibit intermittent behavior as indicated by the wings or ``fat tails'' for large fluctuation values which reflect larger counts compared to a true Gaussian distribution which is indicated by the dashed red curves. Note, though, that the qualitative ``level'' of intermittency decreases with increasing temporal increment, consistent with previous intermittency analyzes on SSX~\cite{schaffner2014a,schaffner2014b}, as well as solar wind results~\cite{bruno2013}.

The increments chosen for the spatial PDFs are $0.46cm$, $0.92cm$, and $1.38cm$, which corresponds to the minimum, double the minimum, and triple the minimum separation possible given the probe array in the SSX. Unlike the temporal data, the spatial data is not believed to probe any as far into dissipation range scales (for reference, an ion inertial length of 0.6cm  and an ion gyroradius of 0.1cm is calculated in this plasma). Thus the spatial data is likely representative of inertial range turbulence and at most transition range. The resulting PDFs support this notion as they tend to have a more Gaussian-like feature for small fluctuations and only slightly elevated tails. There does appear to be a trend toward increasing Gaussian-ness with increasing increment similar to the temporal data. The curves do no appear as smooth as the temporal data though, as the histograms have larger breaks---in particular, the $0.46cm$ increment PDF. This potentially reflects both a lower amount of increment statistics available for the spatial measurement as well as the possible variations in the PDF due to spatial variation. 

As discussed before, the use of PDFs of increments can clearly demonstrate the lack or presence of intermittent behavior in a signal, however in a primarily qualitative way. In order to demonstrate quantitative scaling behavior, and be able to extract more physical interpretations from the data, a structure function analysis is needed and is discussed in the next section.

\section{Structure Functions and Scaling with Order}\label{sec:structs}

To produce a more quantitative metric from the PDFs of increments, one can calculate moments of the histogram to any order $p$, and for each time or spatial delay $\Delta x$. This is equivalent to calculating the structure function in Equation~\ref{eq:structfunc} for the magnetic field increments. Integer values of $p$ correspond to the moments of the PDF ($p=2$ is the second moment, $p=3$ is the third moment, etc.) though given the form of Equation~\ref{eq:structfunc}, $p$ is not restricted to integer values in the structure function formalization.

\begin{figure*}[!htbp]
\centerline{
\includegraphics[width=17cm]{Bmod_timeandspace_StructureFunction100313Shots41to80_forStructFuncpaper.png}}
\caption{\label{fig:structfuncs} }
\end{figure*}

\begin{figure}[!htbp]
\centerline{
\includegraphics[width=8.5cm]{Bmod_timeandspace_Flatness100313Shots41to80_forStructFuncpaper.png}}
\caption{\label{fig:flatness} }
\end{figure}

Figure~\ref{fig:structfuncs} shows the structure functions, $S^{p}(\Delta x)$, for temporal ($\Delta x = \Delta \tau$) data on the left and spatial ($\Delta x = \Delta r$) data on the right. The integer structure functions from $p=1$ to $p=6$ are shown. The structure functions vary as a function of $\tau$ or $r$ which confirms the presence of intermittency in the signal. For the temporal results, it is clear that this variation changes as function of time scale, with steeper slopes at small values of $\tau$ transitioning to shallower slopes at large values of $\tau$. This indicates that the relative degree of intermittency of the signal increases when analyzed at smaller scales. The transition range is also consistent with the transition from the inertial to the dissipation range of the cascade as determined by the change in spectral index of the frequency spectrum. This difference between inertial and dissipation range intermittency suggests that the physical mechanism underlying the dissipation dynamics has a more intermittent nature to it. Explanations for this behavior in the solar wind have typically center around the presence of current sheets in the magnetic turbulence(Osmin PRL 2014). The observation of this difference between dissipation and inertial range fluctuations in this laboratory plasma suggests a similar phenomena~\cite{schaffner2014b}. 

The spatial structure function do not show a variation with scale; however the slopes and thus the degree of intermittency is consistent with the temporal structure functions in the inertial range. Since the spatial measurements can only probe inertial range scales in the plasma, this result supports the distinction between dissipation and inertial range intermittency seen in the temporal data.

It can also be instructive to normalize the structure functions and in particular highlight the ``level'' of non-Gaussianness of a distribution. A normalized structure function can be constructed as,
\begin{equation}
S_{norm}^{p}(\Delta x) = \frac{S^{p}(\Delta x)}{(S^2(\Delta x))^{(p/2)}}
\label{eq:normstructfunc}
\end{equation}
where $p$ is the order of the moment. Using this normalized structure function, a Gaussian timeseries would have a constant value. For example, for $p=4$, this quantity becomes the flatness or kurtosis. Figure~\ref{fig:flatness} shows the flatness for both temporal and spatial. A Gaussian timeseries would have a constant value of 3. Both Figure~\ref{fig:flatness}(a) and (b) show an excursion from a Gaussian distribution at smaller increments, though it is clearly more pronounced for the temporal data.

A final characteristic, self-similarity of the turbulence, can be extracted from the structure function analysis by examining the slope of the structure functions as a function of order, $\zeta(p)$. The red lines in Figure~\ref{fig:structfuncs} are fits to the structure functions. In the temporal data, fits are applied to two separate regions that correspond to the dissipation and inertial regions based on spectral frequency analysis, while the spatial data has only one fit for the entire region. A fit is made in each of these regions for each structure function using order $p$ which ranges between $0.1$ and $10$ in steps of $0.1$. The results of this scan are displayed in Figure~\ref{fig:slopevsmom} for both time and space. The five separate magnetic field measurements are shown now, with inertial range fits indicate by solid lines and dissipation range fits indicate with dashed lines. In this construction, a timeseries that exhibits self-similarity would produce a line with a constant slope. As the increments are raised to increasingly higher powers, the value of the structure function should also increase exponentially. If the signal is self-similar, an increment between two points in the time series at one scale should have on average the same relative increment between two points at a different scale. In other words, a big fluctuation compared to a small fluctuation at one scale should appear to have the same relative ratio at a different scale. Thus, the structure function should change at a constant rate based on this ratio. If the signal is not self-similar, then the ratio is not constant and differences are increasingly accentuated by the raised exponent resulting in a non-linear scaling.

The original 1941 Kolmogorov turbulence theory~\cite{kolmogorov41} in fact predicted a self-similar scaling such that the structure function slope would be equal to one-third of the power. The dashed purple line indicates this prediction in Figure~\ref{fig:slopevsmom}. The inertial range curves for all five temporal magnetic measurements in Figure~\ref{fig:slopevsmom}(a) sit relatively near the Kolmogorov theory line but none of the lines exhibiting completely linear behavior. This indicates that the inertial range fluctuations are not self-similar. The spatial curves in Figure~\ref{fig:slopevsmom}(b) support this observation.

The dissipation range curves, in contrast, are clearly linear. This indicates that the dissipation range fluctuations are self-similar while also having a larger degree of self-similarity than inertial range fluctuations. The Hurst exponents for the dissipation range are consistently about 1 for the various measured B values, steeper than the K41 prediction of H=1/3.



%To see intermittency most clearly, need a normaloized structue function. unnormalized Strcture function demonstrates a distinctive change in the scaling at different time scsales. that is different scaling behavior as a function of tau. we can model the scaling as S = a*tau^\psi

\begin{figure}[!htbp]
\centerline{
\includegraphics[width=8.5cm]{Bfield_StructureFunctionSlope_vs_Moment_timeandspace_100313Shots41to80_forStructFuncpaper.png}}
\caption{\label{fig:slopevsmom} }
\end{figure}

\section{Helicity Scaling}

The structure function analysis is also used to examine the effect of varying the amount of magnetic helicity in the plasma on the intermittency and self-similarity of the magnetic fluctuations. Previous work has shown that the degree of intermittency in fluctuations of $\dot{B}=dt/dB$ (not $B$), as determined by a calculation of the flatness, increases on averaged with an increasing amount of injected magnetic helicity~\cite{schaffner2014b}. This work revisits that result by examining the trend in $B$ using unnormalized structure functions and the resulting relationship between structure function slope and $p$. 

\begin{figure}[!htbp]
\centerline{
\includegraphics[width=8.5cm]{Bmod_StructureFunctionSlope_vs_Moment_helicityscan.png}}
\caption{\label{fig:helscan} }
\end{figure}

Figure~\ref{fig:helscan} shows the slope versus order for eight different helicity states for inertial range fluctuations (solid lines) and dissipation range fluctuations (dashed lines) Recalling that the steepness of the slope is indicative of relative degree of intermittency, the order of the inertial range curves for each helicity indicates increasing intermittency with increasing helicity, again consistent with the findings for an analysis of $\dot{B}$ timeseries~\cite{schaffner2014b}. However, the structure function analysis here further shows that with the exception of the zero helicity state, the inertial range turbulence for any helicity is not self-similar. The dissipation range lines, however, show that the dissipation range intermittency is self-similar regardless of the amount of injected magnetic helicity in the plasma.

\section{Discussion}

The observation of both multifractal and monofractal scaling of magnetic fluctuation structure functions in SSX has implications for both the dissipative nature of the turbulence as well as for universality of MHD turbulence itself. The multifractal scaling model of inertial range turbulence posits that unlike the global scale invariance assumed in Kolmogorov's original 1941 paper, fully-developed turbulence exhibits local scale invariance which allows for a set of scaling exponents, rather than a single one~\cite{benzi1984,frisch1991}. This effect can manifests in the structure function analysis as a non-linear relationship between $\zeta$ and the order, $p$, as in Figures~\ref{slopevsmon} and ~\ref{helscan}. Such multifractal scaling has been previously observed in the velocity increment structure functions of hydrodynamics as well as in the magnetic increment structure functions of solar wind turbulence~\cite{tu1995}. The observation of multifractal scaling in a laboratory MHD turbulence experiment suggests universality in the nature of inertial range MHD intermittency. Similarly, observation of a transition to monoscaling structure functions in dissipative regimes indicates similar scaling transitions may be underway as suggested theoretically~\cite{frisch1991,chevillard2005} and observed in solar wind plasmas~\cite{kiyani2009,kiyani2010,kiyani2013}. This transition model, called an intermediate dissipation range~\cite{frisch1991} or a near-dissipation range~\cite{chevillard2005} predicts that the ''dissipation scale`` is actually dependent on a particular scaling exponent. As each of these ''dissipation scales`` are approached, the contribution to the turbulence of the associated scaling exponent is eliminated, eventually yielding to a small subset of scaling exponents---effectively, monoscaling.  As is noted in Kiyani 2010, however, the multifractal transition model would indicate a gradual transition, while the observed change from multifractal scaling to monofractal scaling in the solar wind, as well in the turbulence reported here, occurs sharply. This, perhaps, is an indicative of the type of dissipation under investigation. While the multifractal model was developed using viscosity in a conventional fluid as the dissipative mechanism, MHD turbulence can have a much more complex dissipation system. Previous work on SSX has suggested that current sheets, at the scale of the ion inertial length, may be a candidate as a dissipation mechanism in the laboratory MHD plasma~\cite{schaffner2014c}. It is perhaps the case that while scaling exponents in conventional fluid turbulence can be gradually terminated by a series of associated viscous scales, that scaling exponents in MHD turbulence might instead be terminated by a single dissipation scale associated with a current sheet layer. It is also intriguing to note the similarity of the scaling transition in the collisionless plasma of the solar wind to the collisional plasma of the laboratory MHD. This supports the notion of current sheets and reconnection as a source of dissipation in both plasmas as these structures are not likely to be affected by the collisionality of the plasma.

%The value of this intermittency and structure function analysis lies in the ability to connect the observed trends to possible physical mechanisms. The most interesting result of this analysis is the observed distinction between inertial and dissipation range intermittency and self-similarity characteristics.

%Mono-fractal scaling was observed in the dissipation regime of the solar wind~\cite{kiyani2009,kiyani2013}; however, even at these small scales, the plasma is considered to be collisionless. The laboratory plasma examined here is quite collisional. This suggests that the underlying physical mechanisms may that cause mono-versus multi-fractal scaling may be independent of collisionality.

%physical connections

\section{Conclusions}\label{sec:conclusions}

% ----------------------------------------------------------------
\section*{Acknowledgements}
  We gratefully acknowledge many useful discussions with William Matthaeus, Greg Howes, Kris Klein, Robert Wicks, Jason TenBarge and Adrian Wan. This work has been funded by DOE OFES and NSF CMSO.  The simulations were performed using the advanced computing resources (Cray XC30 Edison system) at the National Energy Research Scientific Computing Center.
% ----------------------------------------------------------------
\section*{References}
\begin{thebibliography}{99}

\bibitem{frisch1995}
Frisch, U. 1995, {\it Turbulence} (Cambridge: Cambridge Univ. Press)

\bibitem{schaffner2014c} D.A. Schaffner, M.R. Brown and V.S. Lukin. ``Temporal and Spatial Turbulent Spectra of MHD Plasma and an Observation of Variance Anisotropy''. {\it ApJ} {\bf 790}, 126 (2014).

\bibitem{marsch1997} E. Marsch and C.-Y. Tu. ``Intermittency, non-Gaussian statistics and fractal scaling of MHD fluctuations in the solar wind.'' {\it Nonlinear Processes in Geophysics} {\bf 4}, 101-124 (1997).

\bibitem{kiyani2013} K.H. Kiyani, S.C. Chapman, F. Sahraoui, B. Hnat, O. Fauvarque, and Yu.V. Khotyainsev. Enhanced Magnetic Compressibility and Isotropic Scale Invariance at Sub-Ion Larmor Scales in Solar Wind Turbulence. ApJ. {\bf 763} 10 (2013).

\bibitem{kiyani2009} K.H. Kiyani, S.C. Chapman, Yu. V. Khotyaintsev, M.W. Dunlop and F. Sahraoui. ``Global Scale-Invariant Dissipation in Collisionless Plasma Turbulence.'' {\it Phys. Rev. Lett.} {\bf 103} 075006 (2009). 

\bibitem{kiyani2010} K. Kiyani, S. Chapman, Y. Khotyaintsev, M. Dunlop, and F. Sahraoui. ``Fractal dissipation of small-scale magnetic fluctuations in solar wind turbulence as seen by CLUSTER.'' AIP Conf. Proc. 1216, Twelfth International Solar Wind Conference, ed. M. Maksimovic et al. (Melville, NY: AIP), {\bf 136} (2010).

\bibitem{frisch1991} U. Frisch and M. Vergassola. ``A Prediction of the Multifractal Model: the Intermediate Dissipation Range.'' {\it Europhys. Lett.} {\bf 14} 439-444 (1991).

\bibitem{burlaga1991} L.F. Burlaga. ``Intermittent Turbulence in the Solar Wind''. {\it J. Geophys. Res.} {\bf 96} 5847-5851 (1991).

\bibitem{burlaga1992} L.F. Burlaga. ``Multifractal structure of the magnetic field and plasma in recurrent streams at 1AU.'' {\it J. Geophys. Res.} {\bf 97} 4283-4293 (1992).

\bibitem{consolini1996} G. Consolini, M.F. Marcucci and M. Candidi. ``Multifractal Structure of Auroral Electrojet Index Data.'' {\it Phys. Rev. Lett.} {\bf 76} 4082 (1996).

\bibitem{hnat2003} B. Hnat, S.C. Chapman, G. Rowlands, N.W. Watkins, and M.P. Freeman. ``Scaling in long term data sets of geomagnetic indices and solar wind $\varepsilon$ as seen by WIND spacecraft''.{\it Geophys. Rev. Lett.}. {\bf 30} 2174 (2003).

\bibitem{kolmogorov1941} A. N. Kolmogorov, Dokl. Acad. Nauk. SSSR 30, 301, (1941).

\bibitem{frisch1978} U. Frisch, P.-L. Sulem, M. Nelkin. ``A simple dynamical model of intermittent fully developed turbulence''. {\it J. Fluid Mech.} {\bf 87} 719-736 (1978).

\bibitem{meneveau1987} C. Meneveau and K.R. Sreenivasan. ``Simple Multifractal Cascade Model for Fully Developed Turbulence.'' {\it Phys. Rev. Lett.} {\bf 59} 1424 (1987).

\bibitem{kolmogorov1962} A.N. Kolmogorov. ``A refinement of previous hypothesis concerning the local structure of turbulence in viscous incompressible fluid at high Reynolds number.'' {\it J. Fluid Mech.} {\bf 13} 82 (1962).

\bibitem{chevillard2005} L. Chevillard, B. Castaing, and E. Leveque. ``On the rapid increase of intermittency in the near-dissipation range of fully developed turbulence.'' {\it Eur. Phys. J. B} {\bf 45} 561-567 (2005).

\bibitem{pagel2002} C. Pagel and A. Balogh. ``Intermittency in the solar wind: A comparison between solar minimum and maximum using Ulysses Data''. {\it J. Geophys. Res.} {\bf 107} 1178 (2002).

\bibitem{carbone1993} V. Carbone. ``Cascade Model for Intermittency in Fully Developed Magnetohydrodynmaic Turbulence.'' {\it Phys. Rev. Lett.} {\bf 71} 1546 (1993).

\bibitem{biskamp1993} D. Biskamp ``Cascade Model for magnetohydrodynamic turbulence.'' {\it Phys. Rev. E}. {\bf 50} 2702 (1994).

\bibitem{muller2000} W.-C.M$\ddot{\mathrm{u}}$ller and D. Biskamp. ``Scaling Properties of Three-Dimensional Magnetohydrodynamic Turbulence''. {\it Phys. Rev. Lett.} {\bf 84} 475 (2000).

\bibitem{biskamp2000} D. Biskamp and W.-C.M$\ddot{\mathrm{u}}$ller. ``Scaling Properties of three-dimensional isotropic magnetohydrodynamic turbulence.'' {\it Phys. Plasmas}. {\bf 7} 4889 (2000).

\bibitem{boldyrev2002} S. Boldyrev. ``Kolmogorov-Burgers Model for Star-Forming Turbulence.'' {\it Astrophysical J.} {\bf 569} 841-845 (2002).

\bibitem{alexandrova2008} O. Alexandrova, V. Carbone, P. Veltri and L. Sorriso-Valvo. ``Small-Scale Energy Cascade of the Solar Wind Turbulence.'' {\it Astrophysical J.} {\bf 674} 1153-1157 (2008).

\bibitem{bruno2013} R. Bruno and V. Carbone. Living Rev Solar Phys {\bf 10} (2013).







\bibitem{schaffner2014a} D.A. Schaffner {\it et al.} Turbulence analysis of an experimental flux rope plasma. {\bf 56} 064003 (2014).

\bibitem{schaffner2014b} D.A. Schaffner, A. Wan, and M. R. Brown, ``Observation of turbulent intermittency scaling with magnetic helicity in an MHD plasma wind-tunnel'', {\it Phys. Rev. Letters} {\bf 112}, 165001 (2014).



\bibitem{brown14} M.R. Brown and D.A. Schaffner. ``Laboratory sources of turbulent plasma: a unique MHD plasma wind tunnel''. {\it Plasma Sources and Science Technology}. {\bf 23} 063001 (2014).

\bibitem{brown15a} M.R. Brown and D.A. Schaffner. ``SSX MHD plasma wind tunnel''. {\it Journal of Plasma Physics}. (2015).



%\bibitem{alexandrova09} O. Alexandrova, J. Saur, C. Lacombe, A. Mangeney, J. Mitchell, S.J. Schwartz, and P. Robert. Universality of Solar Wind Turbulent Spectrum from MHD to Electron Scales. Phys. Rev. Lett. {\bf 103} 165003 (2009).

%\bibitem{sahraoui09} F. Sahraoui, M.L. Goldstein, P. Robert and Yu. V. Khotyaintsev. Evidence of a Cascade and Dissipation of Solar-Wind Turbulence at the Electron Gyroscale. Phys. Rev. Lett. {\bf 102} 231102 (2009).

%\bibitem{sahraoui06} F. Saharoui, G. Belmont, L. Rezeau, N. Cornilleau-Wehrlin, J.L. Pincon, A. Balogh. Anistropic Turbulent Spectra in the Terrestrial Magnetosheath as Seen by the Cluster Spacecraft. Phys. Rev. Lett. {\bf 96} 075002 (2006).

%\bibitem{yordanova08} E. Yordanova, A. Vaivads, M. Andre, S.C. Buchert and Z. Voros. Magnetosheat Plasma Turbulence and Its Spatiotemporal Evolution as Observed by the Cluster Spacecraft. Phys. Rev. Lett. {\bf 100} 205003 (2008).

%\bibitem{goldstein95} M.L. Goldstein, D.A. Roberts, and W.H. Matthaeus. Magnetohydrodynamic Turbulence in the Solar Wind. Annu. Rev. Astron. Astrophys. {\bf 33} 283-325 (1995).

%\bibitem{tumarsch95} C.-Y. Tu and E. Marsch. MHD Structures, Waves and Turbulence in the Solar Wind: Observations and Theories. Space Sci. Rev. {\bf 73} 1-210 (1995).

%\bibitem{podesta07} J.J. Podesta, D.A. Roberts, and M.L. Goldstein. Spectral Exponents of Kinetic and Magnetic Energy Spectra in Solar Wind Turbulence. ApJ. {\bf 664} 543-548 (2007).

%\bibitem{chen13} C.H.K. Chen, S. Boldyrev, Q. Xia, and J.C. Perez. Nature of Subproton Scale Turbulence in the Solar Wind. Phys. Rev. Lett. {\bf 110} 225002 (2013).

%\bibitem{montgomery81} D. Montgomery and L. Turner. Anistropic magnetohydrodynamic turbulence in a strong external magnetic field. Phys. Fluids {\bf 24} 825 (1981).

%\bibitem{matthaeus90} W.H. Matthaeus, M.L. Goldstein, D.A. Roberts. Evidence for the Presence of Quasi-Two-Dimensional Nearly Incompressible Fluctuations in the Solar Wind. J. Geophys. Res. {\bf 95} 20673 (1990).

%\bibitem{goldreich95} P. Goldreich and S. Sridhar. Toward a Theory of Interstellar Turbulence II. Strong Alvenic Turbulence. ApJ. {\bf 438} 763 (1995).

%\bibitem{zhou04} Y. Zhou, W.H. Matthaeus and P. Dmitruk. Magnetohydrogynamic turbulence and time scales in astrophysical and space plasmas. Rev. Mod. Phys. {\bf 76} 1015 (2004).

%\bibitem{boldyrev06} S. Boldyrev. Spectrum of Magnetohydrodynamic Turbulence. Phys. Rev. Lett. {\bf 96} 115002 (2006).

%\bibitem{dasso05} S. Dasso, L.J. Milano, W.H. Matthaeus, and C.W. Smith. Anistropy in Fast and Slow Solar Wind Fluctuations. ApJ. {\bf 635} L181 (2005).

%\bibitem{horbury08} T.S. Horbury, M. Forman, S. Oughton. Anisotropic Scaling of Magnetohydrodynamic Turbulence. Phys. Rev. Lett. {\bf 101} 175005 (2008).

%\bibitem{podesta09} J.J. Podesta. Dependence of Solar-Wind Power Spectra on the Direction of the Local Mean Magnetic Field. ApJ. {\bf 698} 986 (2009).

%\bibitem{perri09} S. Perri, E. Yordanova, V. Carbone, P. Veltri, L. Sorriso-Valvo, R. Bruno, and M. Andre. Magnetic turbulence in space plasmas: Scale-dependent effects of anisotropy. J. Geophys. Res. {\bf 114} A02102 (2009).

%\bibitem{wicks10} R.T. Wicks, T.S. Horbury, C.H.K. Chen, A.A. Schekochihin. Power and spectral index anisotropy of the entire intertial range of turbulence in the fast solar wind. Mon. Not. R. Astron. Soc. {\bf 407} L31 (2010).

%\bibitem{he11} J.-S. He, E. Marsch, C.-Y. Tu, Q.-G. Zong, S. Yao, and H. Tian. Two-dimensional correlation functions for density and magnetic field fluctuations in magnetosheath turbulence measured by the Cluster spacecraft. {\bf 116} A06207 (2011).

%\bibitem{horbury12} T.S. Horbury, R.T. Wicks, C.H.K. Chen. Anisotropy in Space Plasma Turbulence: Solar Wind Observations. Space Sci Rev. {\bf 172} 325-342 (2012).

%\bibitem{belcher71} J.W. Belcher and Leverett Davis, Jr. Large-Amplitude Alfven Waves in the Interplanetary Medium, 2. J. Geophys. Res. {\bf 76} 3534 (1971).

%\bibitem{smith06} C.W. Smith, B.J. Vasquez and K. Hamilton. Interplanetary magnetic fluctuation anisotropy in the inertial range. J. Geophys. Res. {\bf 111} A09111 (2006).

%\bibitem{coles89} W.A. Coles and J.K. Harmon. Propagation Observations of the Solar Wind Near the Sun. ApJ. {\bf 337} 1023 (1989).

%\bibitem{harmon05} J.K. Harmon and W.A. Coles. Modeling radio scattering and scintillation observations of the inner solar wind using oblique Alfven/ion cyclotron waves. J. Geophys. Res. {\bf 110} A03101 (2005).

%\bibitem{roberts10} D.A. Roberts. Evolution of the spectrum of solar wind velocity fluctuations from 0.3 to 5 AU. J. Geophys. Res. {\bf 115} A121001 (2010).

%\bibitem{robinson71} D.C. Robinson and M.G. Rusbridge. Structure of Turbulence in the Zeta Plasma. Phys. of Fluids. {\bf 14} 2499 (1971).

%\bibitem{liewer85} P.C. Liewer. Measurements of Microturbulence in Tokamaks and Comparisons with Theories of Turbulence and Anomalous Transport. Nuc. Fus. {\bf 25} 543 (1985).

%\bibitem{tynan09} G.R. Tynan, A. Fujisawa, and G. McKee. A review of experimental drift turbululence studies. Plasma Phys. Cont. Fus. {\bf 51} 113001 (2009).

%\bibitem{howes12a} G.G. Howes, D.J. Drake, K.D. Nielson, T.A. Carter, C.A. Kletzing and F. Skiff. Toward Astrophysical Turbulence in the Laboratory. Phys. Rev. Lett. {\bf 109} 255001 (2012).

%\bibitem{ren11} Y. Ren, A.F. Almagri, G. Fiksel, S.C. Prager, J.S. Sarff and P.W. Terry. Experimental Observation of Anisotropic Magnetic Turbulence in a Reversed Field Pinch Plasma. Phys. Rev. Lett. {\bf 107} 195002 (2011).

%\bibitem{deWit13} T. Dudok de Wit, O. Alexandrova, I. Furno, L. Sorriso-Valvo, G. Zimbardo, Methods for Characterising Microphysical Processes in Plasmas, Space Sci. Rev. {\bf 0038-6308}, p. 1-29 (2013).


%\bibitem{sorrisovalvo99}Sorriso-Valvo, L. {\it et al.} Geophys. Res. Lett. {\bf 26}, 1801–1804 (1999).

%\bibitem{sahraoui10} F. Sahraoui, M.L. Goldstein, G. Belmont, P. Canu and L. Rezeau. Three Dimensional Anisotropic {\it k} Spectra of Turbulence at Subproton Scales in the Solar Wind. Phys. Rev. Lett. {\bf 105} 131101 (2010).

%\bibitem{barnes86}C.W. Barnes, J.C. Fernandez, I. Henins, H.W. Hoida, T.R. Jarboe, S.O. Knox, G.J. Marklin, K.F. McKenna. Experimental determination of the conservation of magnetic helicity from the balance between source and spheromak. Phys. Fluids {\bf 29} 3415 (1986).

%\bibitem{zhang11} X. Zhang, D. Dandurand, T. Gray, M.R. Brown, and V.S. Lukin, Calibrated Cylindrical Mach Probe in a Plasma Wind Tunnel, Rev. Sci. Instr. {\bf 82}, 033510 (2011).

%\bibitem{bondeson81} A. Bondeson, G. Markon, Z.G. An, H.H. Chen, Y.C. Lee, and C.S. Liu. Tilting instibility of a cylindrical spheromak. Phys. Fluids. {\bf 24} 1682 (1981).

%\bibitem{jarboe93}T.R. Jarboe. Review of Spheromak Research. Plasma Phys. Cont. Fus. {\bf 36} 945 (1994).

%\bibitem{taylor86} J. B. Taylor, Rev. Mod. Phys. {\bf 58}, 741 (1986).

%\bibitem{clauset09}A. Clauset, C. Rohilla Shalizi, M.E.J. Newman, Power-law distributions in empirical data, SIAM Rev. {\bf 51}, 661703 (2009).

%\bibitem{sarkar14}A. Sarkar, A. Bhattacharjee and F. Ebrahimi. Plasma $\beta$ Scaling of Anisotropic Magnetic Field Fluctuations in the Solar Wind Flux Tube. ApJ. {\bf 783} 65 (2014).

%\bibitem{tenbarge12} J.M. TenBarge, J.J. Podesta, K.G. Klein and G.G. Howes. Interpreting Magnetic Variance Anisotropy Measurements in the Solar Wind. ApJ. {\bf 753} 107 (2012).

%\bibitem{hamilton08} K. Hamilton, C.W. Smith, B.J. Vasquez and R.J. Leamon. Anistropies and helicities in the solar wind intertial and dissipation ranges at 1 AU. J. Geophys. Res. {\bf 113} A01106 (2008).

%\bibitem{matthaeus12}W.H. Matthaeus, S. Servidio, P. Dmitruk, V. Carbone, S. Oughton, M. Wan and K.T. Osman. Local Anisotropy, Higher Order Statistics and Turbulence Spectra. ApJ. {\bf 750}, 103 (2012).

%\bibitem{torrence98}C. Torrence, G.P. Compo, A practical guide to wavelet analysis. Bull. Am. Meteorol. Soc. {\bf 79}, 6178 (1998).

%\bibitem{klein12} K.G. Klein, G.G. Howes, J.M. TenBarge, S.D. Bale, C.H.K. Chen and C.S. Salem. Using Synthetic Spacecraft Data to Interpret Compressible Fluctuations in Solar Wind Turbulence. ApJ. {\bf 755} 159 (2012).

%\bibitem{chen12} C.H.K. Chen, C.S. Salem, J.W. Bonnell, F.S. Mozer and S.D. Bale. Density Fluctuation Spectrum of Solar Wind Turbulence between Ion and Electron Scales. Phys. Rev. Lett. {\bf 109} 035001 (2012).

%\bibitem{Belmabrouk98} Belmabrouk, H., and M. Michard (1998), Taylor length scale measurement by laser Doppler velocimetry, Exp. Fluids, 25, 69Ð76.

%\bibitem{Matthaeus05} Matthaeus, W. H. and Dasso, S. and Weygand, J. M. and Milano, L. J. and Smith, C. W. and Kivelson, M. G., Phys. Rev. Lett. 95, 231101 (2005) Spatial Correlation of Solar-Wind Turbulence from Two-Point Measurements

%\bibitem{Weygand07} Weygand, J. M., Matthaeus, W. H., Dasso, S., Kivelson, M. G., and Walker, R. J. (2007), J. Geophys. Res., 112, A10201.

%\bibitem{Weygand09} Weygand, J. M., Matthaeus, W. H., Dasso, S., Kivelson, M. G., Kristler, L. M., and Mouikis, C. (2009), J. Geophys. Res., 114, A07213.

%\bibitem{Weygand10} Weygand, J. M., Matthaeus, W. H., El-Alaoui, M., Dasso, S., and Kivelson, M. G. (2010), J. Geophys. Res., textit115, A12250.

%\bibitem{Weygand11} Weygand, J. M., Matthaeus, W. H., Dasso, S., and Kivelson, M.G. (2011), J. Geophys. Res., 116, A08120.

%\bibitem{Matthaeus08} Matthaeus W. H., Weygand, J. M., Chuychai, P., Dasso, S., Smith, C. W., and Kivelson, M. (2008), Astrophys. J., 678, L141.

%\bibitem{frisch95}Frisch, U. 1995, {\it Turbulence} (Cambridge: Cambridge Univ. Press)

%\bibitem{sorrisovalvo99}Sorriso-Valvo, L. {\it et al.} Geophys. Res. Lett. {\bf 26}, 1801–1804 (1999).

%\bibitem{wan12}Wan, M. {\it et al.} ApJ. {\bf 744} 177 (2012).

%\bibitem{sorrisovalvo01}Sorriso-Valvo, L. {\it et al.} Planet. Space Sci. {\bf 49}, 1193–1200 (2001).

%\bibitem{marrelli05}Marrelli, L. {\it et al.} Phys. Plasmas. {\bf 12}, 030701 (2005).

%\bibitem{Greco08}A. Greco, P. Chuychai, W. H. Matthaeus, S. Servidio and P. Dmitruk, Intermittent MHD structures and classical discontinuities, Geophys. Res. Lett. {\bf 35}, L19111 (2008).

%\bibitem{Greco09}Greco, A., Matthaeus, W. H., Servidio, S., Chuychai, P., and Dmitruk, P.: Statistical Analysis of Discontinuities in Solar Wind ACE Data and Comparison with Intermittent MHD Turbulence, ApJ {\bf 691}, L111 (2009).

%\bibitem{Wan09}Wan, M., Oughton, S., Servidio, S., and Matthaeus, W. H.: Generation of non-Gaussian statistics and coherent structures in ideal magnetohydrodynamics, Phys. Plasmas {\bf 16}, 080703 (2009).

%\bibitem{Servidio11b}Servidio, S. {\it et al}, J. Geophys. Res. {\bf 116}, A09102 (2011).

%\bibitem{Gray13} T. Gray, M. R. Brown, and D. Dandurand. Phys. Rev. Lett. {\bf 110}, 085002 (2013). 

%\bibitem{Taylor86} J. B. Taylor, Rev. Mod. Phys. {\bf 58}, 741 (1986).

%\bibitem{Matthaeus80} W.H. Matthaeus and D. Montgomery, Ann. N.Y. Acad. Sci. {\bf 357}, 203 (1980).

%\bibitem{wan12}M. Wan, K. T. Osman, W. H. Matthaeus, and S. Oughton, Investigation of intermittency in magnetohydrodynamics and solar wind turbulence: scale-dependent kurtosis, ApJ {\bf 744}, 171 (2012).

%\bibitem{Gray10}T. Gray, V. S. Lukin, M. R. Brown, C. D. Cothran, Three-dimensional reconnection and relaxation of merging spheromak plasmas, Phys. Plasmas {\bf 17}, 102106 (2010).

%\bibitem{goldstein94}Goldstein, M.L., Roberts, D.A. and Fitch, C.A. Jour. Geo. Res. {\bf 99} 11519-11538 (1994).

%\bibitem{ji95}Ji, H., Prager, S.C. and Sarff, J.S. Phys. Rev. Lett. {\bf 74} 2945 (1995).

%\bibitem{telloni12}Telloini, D. {\it et al.}. ApJ. {\bf 751} 19 (2012).

%\bibitem{matthaeusVelli11}Matthaeus, W.H. and Velli, M. Space Sci. Rev. {\bf 160} 145-168 (2011).

%\bibitem{greco12}A. Greco {\it et al.} ApJ. {\bf 749} 105 (2012).

\end{thebibliography}

\end{document}

